\chapter{Исследовательский раздел}

В данном разделе будут представлены технические характеристики и будет проведено исследование для определения зависимости времени работы от сложности запроса в базу данных.

\section{Технические характеристики}
Технические характеристики устройства, на котором выполнялось исследование:
\begin{itemize}
	\item операционная система: Ubuntu 20.04~\cite{ubuntu};
	\item размер оперативной памяти: 16 Гбайт;
	\item процессор AMD Ryzen 5 5500U with Radeon Graphics.
\end{itemize}
	
На протяжении всего тестирования компьютер был подключен к сети питания.
\section{Исследование}
Задача заключалась в исследовании зависимости времени работы от сложности запроса.
Исследование проводилось на заранее заготовленной таблице, количество записей в которой было равно 500 штук.

Градация сложности запросов была следующая:
\begin{enumerate}
	\item запрос первой сложности:
		\includelisting
		{1.sql}
		{Запрос первой сложности}
	\item запрос второй сложности:
		\includelisting
		{2.sql}
		{Запрос второй сложности}
	\item запрос третьей сложности:
		\includelisting
{3.sql}
{Запрос третьей сложности}
	\item запрос четвертой сложности:
		\includelisting
{4.sql}
{Запрос четвертой сложности}
	\item запрос пятой сложности:
		\includelisting
{5.sql}
{Запрос пятой сложности}
	\item запрос шестой сложности:
		\includelisting
{6.sql}
{Запрос шестой сложности}
\end{enumerate}  

Для каждого запроса время замерялось 1000 раз и суммировалось.
После бралось среднее значение.

По итогам исследования получились следующие результаты, представленные в таблице~\ref{table:res} и на рисунке~\ref{img:chart}.

\begin{table}[h]

	\centering
	\begin{tabular}{ | c | c | }
		\hline
		Уровень сложности запроса & Время работы, мс \\ \hline
		1 & 1.181 \\
		2 & 1.383 \\
		3 & 1.397 \\
		4 & 1.402 \\
		5 & 1.445 \\
		6 & 1.532 \\
		\hline
	\end{tabular}
	\caption{Результаты исследования}
				\label{table:res}
\end{table}



\includeimage
{chart} % Имя файла без расширения (файл должен быть расположен в директории inc/img/)
{f} % Обтекание (с обтеканием)
{H} % Положение рисунка (см. wrapfigure из пакета wrapfig)
{\textwidth} % Ширина рисунка
{График зависимости времени работы от сложности запроса} % Подпись рисунка


\section*{Вывод}
В ходе выполнения исследовательской части было выявлено, что время работы напрямую зависит от сложности запроса --- чем сложнее запрос, тем больше времени системе требуется на его обработку.
Также на графике можно наблюдать довольно резкий скачок времени работы в 1.16 раз между Запросом 1 и Запросом 2.
Это можно объяснить тем, что в инструкции SELECT Запроса 1 необходимо вернуть одно значение, а в инструкции SELECT Запроса 2 4 значения.
При проходе по всей таблице, системе необходимо вернуть ответ, который будет включать в себя больше значений, чем при Запросе 1.
Также большое время при выполнении Запроса 6 можно объяснить работой с двумя таблицами.

 