

\begin{appendices}
	\chapter{Тестирование}

\begin{lstlisting}[language=go, label=lst:testing_code]
func TestEquipmentService_Get(t *testing.T) {
type fields struct {
	equipmentRepo _interface.IEquipmentRepository
	reserveRepo   _interface.IReserveRepository
}
type args struct {
	request *dto.GetEquipmentRequest
}
tests := []struct {
	name          string
	fields        fields
	args          args
	wantEquipment *model.Equipment
	wantErr       bool
}{
{
	name: "test_pos_01",
	args: args{
		&dto.GetEquipmentRequest{
			Id: 1,
		},
	},
	wantErr: false,
	wantEquipment: &model.Equipment{
		Id:            1,
		Name:          "1",
		StudioId:      1,
		EquipmentType: 1,
	},
},
{
	name: "test_neg_01",
	args: args{
		&dto.GetEquipmentRequest{
			Id: 0,
		},
	},
	wantErr:       true,
	wantEquipment: nil,
	},
}
for _, tt := range tests {
	prodRepo := new(mocks.IEquipmentRepository)
	prodRepo.On("Get", context.Background(), tt.args.request).Return(&model.Equipment{
		Id:            1,
		Name:          "1",
		StudioId:      1,
		EquipmentType: 1,
	}, nil)
	t.Run(tt.name, func(t *testing.T) {
		s := EquipmentService{
			equipmentRepo: prodRepo,
		}
		gotEquipment, err := s.Get(tt.args.request)
		if (err != nil) != tt.wantErr {
			t.Errorf("Get() error = %v, wantErr %v", err, tt.wantErr)
			return
		}
		if !reflect.DeepEqual(gotEquipment, tt.wantEquipment) {
			t.Errorf("Get() gotEquipment = %v, want %v", gotEquipment, tt.wantEquipment)
		}
	})
}

\end{lstlisting}
\caption{Часть кода модульного тестирования}




\begin{lstlisting}[language=go, label=lst:testing_code]
func TestStudioPostrgresql_Get(t *testing.T) {
type fields struct {
	db *pgxpool.Pool
}

type args struct {
	ctx     context.Context
	request *dto.GetStudioRequest
}

tests := []struct {
	name string
	args       args
	wantStudio *model.Studio
	wantErr    bool
}{
	{
		name: "test_pos_01",
		args: args{
			ctx: context.Background(),
			request: &dto.GetStudioRequest{
			Id: 1,
			},
		},
		wantStudio: &model.Studio{
			Id:   1,
			Name: "first",
		},
		wantErr: false,
	},
}

for _, tt := range tests {
	t.Run(tt.name, func(t *testing.T) {
		p := postgresql.NewStudioRepository(testDbInstance)

		gotStudio, err := p.Get(tt.args.ctx, tt.args.request)
		if (err != nil) != tt.wantErr {
			t.Errorf("Get() error = %v, wantErr %v", err, tt.wantErr)
		return
		}
		if !reflect.DeepEqual(gotStudio, tt.wantStudio) {
			t.Errorf("Get() gotStudio = %v, want %v", gotStudio, tt.wantStudio)
		}
	})
}
	\end{lstlisting}
	\captionof{figure}{Часть кода интеграционного тестирования}


\begin{lstlisting}[language=go, label=lst:testing_res]
=== RUN   TestEquipmentRepository_GetFullTimeFreeByStudioAndType
--- PASS: TestEquipmentRepository_GetFullTimeFreeByStudioAndType (0.01s)
=== RUN   TestEquipmentRepository_GetFullTimeFreeByStudioAndType/test_pos_01
	--- PASS: TestEquipmentRepository_GetFullTimeFreeByStudioAndType/test_pos_01 (0.01s)
=== RUN   TestEquipmentRepository_GetNotFullTimeFreeByStudioAndType
--- PASS: TestEquipmentRepository_GetNotFullTimeFreeByStudioAndType (0.01s)
=== RUN   TestEquipmentRepository_GetNotFullTimeFreeByStudioAndType/test_pos_01
	--- PASS: TestEquipmentRepository_GetNotFullTimeFreeByStudioAndType/test_pos_01 (0.01s)
=== RUN   TestReserveRepository_Add
--- PASS: TestReserveRepository_Add (0.00s)
=== RUN   TestReserveRepository_GetByRoomId
--- PASS: TestReserveRepository_GetByRoomId (0.01s)
=== RUN   TestReserveRepository_GetByRoomId/test_pos_01
	--- PASS: TestReserveRepository_GetByRoomId/test_pos_01 (0.01s)
=== RUN   TestReserveRepository_IsRoomReserve
--- PASS: TestReserveRepository_IsRoomReserve (0.00s)
=== RUN   TestReserveRepository_IsRoomReserve/test_pos_01
	--- PASS: TestReserveRepository_IsRoomReserve/test_pos_01 (0.00s)
=== RUN   TestReserveRepository_IsRoomReserve/test_pod_02
	--- PASS: TestReserveRepository_IsRoomReserve/test_pod_02 (0.00s)
=== RUN   TestReserveRepository_IsEquipmentReserve
--- PASS: TestReserveRepository_IsEquipmentReserve (0.00s)
=== RUN   TestReserveRepository_IsEquipmentReserve/test_pos_01
	--- PASS: TestReserveRepository_IsEquipmentReserve/test_pos_01 (0.00s)
=== RUN   TestRoomRepository_GetByStudio
--- PASS: TestRoomRepository_GetByStudio (0.00s)
=== RUN   TestRoomRepository_GetByStudio/test_pos_01
	--- PASS: TestRoomRepository_GetByStudio/test_pos_01 (0.00s)
=== RUN   TestStudioPostrgresql_Get
--- PASS: TestStudioPostrgresql_Get (0.00s)
=== RUN   TestStudioPostrgresql_Get/test_pos_01
	--- PASS: TestStudioPostrgresql_Get/test_pos_01 (0.00s)
=== RUN   TestStudioRepository_Update
--- PASS: TestStudioRepository_Update (0.04s)
=== RUN   TestStudioRepository_Update/test_pos_01
	--- PASS: TestStudioRepository_Update/test_pos_01 (0.04s)
=== RUN   TestStudioRepository_Add
--- PASS: TestStudioRepository_Add (0.11s)
=== RUN   TestStudioRepository_Add/test_pos_01
	--- PASS: TestStudioRepository_Add/test_pos_01 (0.11s)
=== RUN   TestStudioRepository_Delete
--- PASS: TestStudioRepository_Delete (0.02s)
=== RUN   TestStudioRepository_Delete/test_pos_01
	--- PASS: TestStudioRepository_Delete/test_pos_01 (0.02s)
=== RUN   TestUserRepository_GetByLogin
--- PASS: TestUserRepository_GetByLogin (0.01s)
=== RUN   TestUserRepository_GetByLogin/test_neg_01
	--- PASS: TestUserRepository_GetByLogin/test_neg_01 (0.01s)
PASS
\end{lstlisting}
\caption{Результат тестирования}
\chapter{Презентация}
\begin{figure}[h]
	\begin{center}
		\includegraphics[page=1, width=\linewidth]{presentation.pdf}
	\end{center}
	\caption{Презентация -- Слайд 1}
\end{figure}

\begin{figure}[h]
	\begin{center}
		\includegraphics[page=2, width=\linewidth]{presentation.pdf}
	\end{center}
	\caption{Презентация -- Слайд 2}
\end{figure}

\begin{figure}[h]
	\begin{center}
		\includegraphics[page=3, width=\linewidth]{presentation.pdf}
	\end{center}
	\caption{Презентация -- Слайд 3}
\end{figure}

\begin{figure}[h]
	\begin{center}
		\includegraphics[page=4, width=\linewidth]{presentation.pdf}
	\end{center}
	\caption{Презентация -- Слайд 4}
\end{figure}

\begin{figure}[h]
	\begin{center}
		\includegraphics[page=5, width=\linewidth]{presentation.pdf}
	\end{center}
	\caption{Презентация -- Слайд 5}
\end{figure}

\begin{figure}[h]
	\begin{center}
		\includegraphics[page=6, width=\linewidth]{presentation.pdf}
	\end{center}
	\caption{Презентация -- Слайд 6}
\end{figure}

\begin{figure}[h]
	\begin{center}
		\includegraphics[page=7, width=\linewidth]{presentation.pdf}
	\end{center}
	\caption{Презентация -- Слайд 7}
\end{figure}

\begin{figure}[h]
	\begin{center}
		\includegraphics[page=8, width=\linewidth]{presentation.pdf}
	\end{center}
	\caption{Презентация -- Слайд 8}
\end{figure}

\begin{figure}[h]
	\begin{center}
		\includegraphics[page=9, width=\linewidth]{presentation.pdf}
	\end{center}
	\caption{Презентация -- Слайд 9}
\end{figure}

\begin{figure}[h]
	\begin{center}
		\includegraphics[page=10, width=\linewidth]{presentation.pdf}
	\end{center}
	\caption{Презентация -- Слайд 10}
\end{figure}

\begin{figure}[h]
	\begin{center}
		\includegraphics[page=11, width=\linewidth]{presentation.pdf}
	\end{center}
	\caption{Презентация -- Слайд 11}
\end{figure}

\begin{figure}[h]
	\begin{center}
		\includegraphics[page=12, width=\linewidth]{presentation.pdf}
	\end{center}
	\caption{Презентация -- Слайд 12}
\end{figure}

\end{appendices}

