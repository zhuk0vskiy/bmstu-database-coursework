
\chapter{Аналитический раздел}
В данном разделе будет представлен анализ предметной области, проведена формализация задачи, проведена формализация и описание информации для хранения в БД, проведена формализация и описание пользователей и проанализированы модели баз данных. 
\section{Анализ предметной области}
Развитие информационных технологий затронуло различные сферы по всему миру, включая музыкальную.
Музыкальная индустрия одной из первых испытала на себе изменения, связанные с новыми технологиями, и за последние 20 лет она изменилась на всех уровнях~\cite{music_and_it}.

Российский рынок услуг звукозаписи является динамично развивающейся отраслью, которая с каждым годом становится всё более популярной и значимой.
В 2017 году в России насчитывалось около 656 студий звукозаписи, а к 2023 году их количество возросло до примерно 1160 студий и репетиционных точек~\cite{music_stat}.

\includeimage
{stat} % Имя файла без расширения (файл должен быть расположен в директории inc/img/)
{f} % Обтекание (с обтеканием)
{H} % Положение рисунка (см. wrapfigure из пакета wrapfig)
{0.8\textwidth} % Ширина рисунка
{Количество студий звукозаписи на момент 20.04.2023 \cite{music_stat}} % Подпись рисунка



В связи с этим, онлайн бронирование в студии звукозаписи приобрели весомое значение в нынешнее время.

\section{Формализация задачи}
Для выполнения поставленной цели необходимо разработать БД для хранения информации о студиях, об их составляющем, о пользователях и о бронях, которые пользователи создают.

Также необходимо спроектировать и разработать приложение, которое будет предоставлять интерфейс для работы с БД и давать возможность для каждого авторизованного пользователя создавать бронь на определенное время, резервируя комнату, оборудование, продюсера и инструменталиста.

Нужно предусмотреть возможность добавления, изменение и удаление студий, комнат, оборудования, продюсеров и инструменталистов.
Необходимо реализовать разный функционал для разных категорий пользователей.

\section{Формализация и описание информации, подлежащей хранению в БД}
Разрабатываемая БД для приложения бронирования студий должна содержать информацию:
\begin{itemize}
	\item о зарегистрированных пользователях;
	\item об имеющихся студиях;
	\item о комнатах, принадлежащих студиям;
	\item об оборудовании, принадлежащем студиям;
	\item о продюсерах, работающих на студиях;
	\item об инструменталистах, работающих на студиях;
	\item о бронях на выбранное время на определенную комнату, оборудование, продюсера и инструменталиста.
\end{itemize}

На рисунке \ref{img:er} представлена ER--диаграмма сущностей в нотации Чена.

\begin{sidewaysfigure}
	\includeimage
	{er} % Имя файла без расширения (файл должен быть расположен в директории inc/img/)
	{f} % Обтекание (с обтеканием)
	{H} % Положение рисунка (см. wrapfigure из пакета wrapfig)
	{0.85\textwidth} % Ширина рисунка
	{ER--диаграмма в нотации Чена} % Подпись рисунка
\end{sidewaysfigure}
% \caption{ER--диаграмма в нотации Чена}

\newpage

\section{Формализация и описание пользователей проектируемого приложения в БД}
Для взаимодействия с приложением было выделено три категории пользователей:
\begin{enumerate}
	\item гость;
	\item клиент;
	\item администратор.
\end{enumerate}

Гость имеет право воспользоваться только начальным функционалом приложения: просмотром броней, регистрацией и входом в аккаунт.
При успешном прохождении авторизации пользователь автоматически становится авторизованным пользователем.

Функционал клиента является более расширенным и включает в себя: создание, просмотр и отмена уже созданных броней. 
Также есть возможность изменение личных данных, выхода из профиля и выхода из приложения.

Если пользователь войдет под именем администратора, то он будет иметь возможность:
\begin{itemize}
	\item добавления, изменения и удаления студий;
	\item добавления, изменения и удаления комнат;
	\item добавления, изменения и удаления оборудования;
	\item добавления, изменения и удаления продюсеров;
	\item добавления, изменения и удаления инструменталистов.
\end{itemize}
		
На рисунке \ref{img:use_case} представлены пользовательские сценарии.


\includeimage
{use_case} % Имя файла без расширения (файл должен быть расположен в директории inc/img/)
{f} % Обтекание (с обтеканием)
{H} % Положение рисунка (см. wrapfigure из пакета wrapfig)
{0.9\textwidth} % Ширина рисунка
{Пользовательские сценарии} % Подпись рисунка

% В рамках данной работы необходимо разработать веб приложение, 


\section{Анализ существующих баз данных}
Базы данных по способу хранения разделяют на две группы --- колоночные и строковые.
Каждая из них используется в разных целях, но наибольшей популярностью пользуются колоночные базы данных.

\subsection{Колоночные базы данных}
В колоночных базах данных все значения одного атрибута сохраняются последовательно. На диске каждая колонка разделена на блоки фиксированного размера.
В каждом блоке есть заголовок, занимающий пренебрежительно мало места по сравнению с общим размером блока, и сами данные.
Каждой записи в столбце соответствует её позиция (номер строки)~\cite{strokovie_and_kolonochnie_bd}.
\subsection{Строчные базы данных}
Строчное хранение данных обычно означает, что каждый кортеж таблицы сохраняется как единое целое, где значения атрибутов идут одно за другим.
После последнего атрибута одного кортежа за ним следует следующий кортеж той же таблицы.
План запроса представлен в виде дерева, в котором каждый узел имеет одного родителя и несколько дочерних узлов~\cite{strokovie_and_kolonochnie_bd}.


\section{Анализ моделей баз данных}
Модель данных --- совокупность структур данных и операций по их обработке \cite{dbms}.

Существуют модели данных следующих типов:
\begin{itemize}
	\item дореляционные;
	\item реляционные;
	\item постреляционные.
\end{itemize}

\subsection{Дореляционные модели}
К дореляционным моделям относят модель инвертированных списков, иерархическую модель и сетевую модель:

\begin{enumerate}
	\item БД, построенная на основе модели инвертированных списков, состоит из множества файлов, содержащих записи.
	Эти записи в файлах расположены в определенном порядке, который зависит от физической организации данных.
	Для каждого файла можно задать несколько различных упорядоченностей на основе значений некоторых полей записей, обычно с помощью индексов.
	В такой модели данных отсутствуют встроенные ограничения целостности.
	Все ограничения на допустимые данные накладываются программами, работающими с базой данных.
	Одно из немногих возможных ограничений --- это ограничение уникальности, которое задается уникальным индексом;
	
	\item иерархическая модель данных основывается на иерархии типов объектов, где один из типов является главным, а остальные, расположенные на более низких уровнях, являются подчиненными.
	Между главным объектом и подчиненными устанавливается взаимосвязь <<один~ко~многим>>~\cite{dbms};
	
	\item В сетевой модели данных любой объект может быть как главным, так и подчиненным. Один и тот же объект может одновременно быть и владельцем, и членом набора.
	Это означает, что любой объект может участвовать в любом количестве взаимосвязей~\cite{dbms}.
\end{enumerate}

\subsection{Реляционные модели}
База данных, построенная на реляционной модели, представляет собой набор таблиц (отношений).
Эти таблицы и операции над ними составляют реляционную алгебру.
В реляционную алгебру входят такие операции, как проекция, выборка, объединение, пересечение, вычитание, соединение и деление~\cite{dbms}.
Классическая реляционная модель предполагает, что данные в полях записей таблиц неделимы.
Это означает, что информация в таблице представлена в первой нормальной форме. Однако, это ограничение может препятствовать эффективной реализации некоторых приложений.

\subsection{Постреляционные модели}
Постреляционная модель данных представляет собой расширенную версию реляционной модели, которая снимает ограничение неделимости данных в полях записей таблиц.
В этой модели допускаются многозначные поля, содержащие значения, состоящие из подзначений. Набор значений многозначных полей рассматривается как отдельная таблица, встроенная в основную таблицу~\cite{voroneg}.
 
\section*{Вывод}
В данном разделе была проанализирована предметная область, проведена формализация задачи, проведена формализация и описание информации, проведена формализация и описание пользователей и проанализированы модели баз данных. 
